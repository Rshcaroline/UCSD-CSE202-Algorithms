\newpage
\problem{3: Largest set of indices within a given distance} % {10+10+10+20=50}

\problemdes

You are given a sequence of numbers $a_{1}, \dots, a_{n}$ in an array. You are also given a number k. Design an efficient algorithm to determine the size of the largest subset $L \subseteq\{1,2, \cdots, n\}$ of indices such that for all $i, j \in L$ the difference between $a_i$ and $a_j$ is less than or equal to $k$. There is an $O(n \log n)$ algorithm for this problem.
For example, consider the sequence of numbers $a_1 = 7,a_2 = 3,a_3 = 10,a_4 = 7,a_5 = 8,a_6 = 7,a_7 = 1,a_8 = 15,a_9 = 8$ and let $k = 3$. $L = \{1,3,4,5,6,9\}$ is the largest such set of indices. Its size is 6.

\solution

\subsolution{High-level description}

Since we only need to determine the size of the largest subset L instead of determining the subset L itself, indexes are not so important then. What we really care about is the differences between elements and the max differences should be less than or equal to $k$. 

Thus, we can sort this array first, and then iterate through the sorted array, find out the maximum length of continuous elements with a maximum difference less than or equal to $k$. 

One tricky thing is how to find the maximum length efficiently. Our solution is using two pointers, the left one stays at index $i$ and the right one points to index $j$. We have $i<j, A[i]<=A[j]$. 

\begin{itemize}
\item While $A[j]-A[i]<=k$, we shift our right pointer. 
\item Compare the current length $j-i$ with the longest size we've seen before, and take the maximum of these two values.
\item Shift the left pointer to $i+1$, since our array is sorted, then $A[i+1]>=A[i]$, which means $A[j]-A[i+1]<=k$ still stands. However this time the length is $j-(i+1)$, and it must be less than the maximum value we just got. Hence, we can shift our right pointer to find the newest index $j'$.
\item Go to Step 1.
\end{itemize}

After iterate through the whole array, we will get the maximum length $size$.

\subsolution{Pseudo Code}

\begin{algorithm}
  \caption{Largest set of indices within a given distance}
  \KwIn{$A$ as an array, $k$}
  \KwOut{$size$}
  sort the input array $A$\;
  initialize size as the min value of the integer, say, $-\inf$\;
  $leftPointer = 1, rightPointer = 1$\;
  \While{$rightPointer <= len(A)$}
  {
    \While{$A[rightPointer]-A[leftPointer]<=k$}
    {
    	$rightPointer = rightPointer + 1$\;
    }
    $size = \max(size, rightPointer-leftPointer)$\;
    $leftPointer = leftPointer +1$\;
}
    return $size$\;
\end{algorithm}

\subsolution{Correctness}

\subsolution{Time complexity}

The sorting algorithm can take $O(n\log n)$ time complexity. As for the remaining part, for both left pointer and right pointer, we only need to iterate through the array once, hence it's $O(n)$ time complexity.

Overall, it's $O(n\log n)$.



