\newpage
\problem{2: Maximum difference in an array} % {10+10+10+20=50}

\problemdes

Given an array $A$ of integers of length $n$, find the maximum value of $A(j) - A(i)$ over all choices of indexes such that $j > i$.

\solution

\subsolution{High-level description}

At first glance, I thought we could use a sort algorithm and gets $max - min$ as result. But wait, the problem description says \textit{the indexes should be $j > i$}, which means \textit{Order matters}. 

What we really need to do is to record the minimum value $min$ and $maxdifference$ we've ever seen so far. When at a new position $i$, we compare current value $A(i)$ with $min$, if $A(i) < min$, then swatch their values; if $A(i) > min$, then compare $A(i) - min$ with $maxdifference$.

After going over $n$ element of $A$, the final $maxdifference$ can be the maximum difference in an array.

\subsolution{Pseudo Code}

\begin{algorithm}[]
  \caption{Maximum difference in an array}
  \KwIn{$A[1 \dots n]$}
  \KwOut{$maxdifference$ as maximum difference}
  let $min = A[1]$\;
  let $maxdifference = 0$\;
  \For{$i=2; i \le n; i++$}
  {
    \If{$A[i] < min$}
    {
        $min = A[i]$\;
    }
    \If{$maxdifference < A[i] - min$}
    {
        $maxdifference = A[i] - min$\;
    }
  }
  return $maxdifference$\;
\end{algorithm}

\subsolution{Correctness}

The instructor answered on \href{https://piazza.com/class/k10yypdn1hf7ld?cid=11}{Pizza} that we may assume that $n>1$. 

The base case is $n=2$ and there are three cases: $A[1] < A[2]$, $A[1] = A[2]$ and $A[1] > A[2]$. As the algorithm goes, we first initialize $min=A[1], maxdifference=0$. 

\textbf{Case 1:} $A[1] < A[2]$. Then we'll skip the first \textit{if} in the \textit{for} loop and then return $maxdifference=A[2]-A[1]$.

\textbf{Case 2:} $A[1] = A[2]$. Then we'll skip the two \textit{if} in the \textit{for} loop and return $maxdifference=0$.

\textbf{Case 3:} $A[1] > A[2]$. Then we'll update $min=A[2]$ and skip the second \textit{if} in the \textit{for} loop. We return $0$ as result.

All three cases are correct for base case.

If the case is correct for $n$. Then for $n+1$, there are three cases: $A[n+1] < min$, $A[n+1] = min$ and $A[n+1] > min$. 

\textbf{Case 1:} $A[n+1] < min$. Then we'll update $min=A[n+1]$. We only have $n+1$ elements, so there won't be a $s>n+1$, we get the maximum value of $A(s)-A(n+1)$. Thus, we return $maxdifference$ of $A[1...n]$.

\textbf{Case 2:} $A[n+1] = min$. Same as Case 1, we still return $maxdifference$ of $A[1...n]$.

\textbf{Case 3:} $A[n+1] > min$. If $maxdifference<A[n+1]-min$, then we update $maxdifference$ and return it. Else, we return $maxdifference$ of $A[1...n]$.

All three cases are correct for $n$ and $n+1$. By induction, this algorithm is corrrect.



\subsolution{Time complexity}

The operation in the \textit{For} loop can be completed in $O(1)$ time, thus in $O(n)$ iterations the algorithm results in a maximum difference.

\subsolution{Data Structure}

Array is enough. We are able to find the $A[i]$ in $O(1)$ time.



