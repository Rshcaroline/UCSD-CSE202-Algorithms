\newpage
\problem{2: Sorted matrix search} % {10+10+10+20=50}

\problemdes

Given an $m \times n$ matrix in which each row and column is sorted in ascending order, design an algorithm to find an element.

\solution

\subsolution{High-level description}

The basic brute force solution is iterate through each row and column, compare the value of the current element with the target value. However, this takes O(mn) time complexity. 

What we should do is taking advantage of the condition that each row and column is sorted in ascending order, which means we can start iterating from the corner elements and follow a zigzag path to find the target value. 

\subsolution{Pseudo Code}

\begin{algorithm}
  \caption{Sorted matrix search}
  \KwIn{$matrix$ as a two-dimensional array, $target$}
  \KwOut{indexes of the element}
  \If{$len(matrix)$ == 0}
  {
  	$//$ invalid input, return an empty list\;
  	return $0$\;
  }
  $//$ initialize indexes, starting from the bottom left corner\;
  $colIndex, rowIndex = 1, len(matrix)$\;
  $//$ find the rough row index\;
  \While{$rowIndex >= 0$ and $target < matrix[rowIndex][colIndex]$}
  {
  	$rowIndex = rowIndex-1$\;
  }
  $//$ as long as column index is valid\;
  \While{$colIndex <= len(matrix[0])$}
  {
    \If{target == matrix[rowIndex][colIndex]}
    {
       $//$ found the target element\;
       return $(rowIndex, colIndex)$\;
     }
      \If{$target > matrix[rowIndex][colIndex]$}
    {
    	$//$ since the column is sorted in ascending order\;
    	$colIndex = colIndex+1$\;
     }
     $//$ if $target < matrix[rowIndex][colIndex]$\;
     \ElseIf{$rowIndex > 0$}
     {
     	$rowIndex = rowIndex-1$\;
     }
     $//$ row index is invalid\;
     \Else{return $0$\;}
     
     $//$ column index is invalid\;
     return $0$\;
}
\end{algorithm}

\subsolution{Correctness}

The element we want to find is $x$. If x is in the matrix, let's prove $x$ must be in the submatrix whose bottom left corner is $y$.

1. This clearly holds in the beginning, since our algorithm starts from the bottom left corner and $y$ is the bottom left corner of the input matrix. It will output $y$'s indexes.

2. If $y>x$, then all elements from $y$ to the right end are also larger than $x$, so $x$ cannot be contained in that row. Therefore our assumption continues to hold with $x$ replaced with its neighbor below (if any. If $y$ has no below neighbors, we can deduce that $x$ is not found in the matrix). 

3. If $y<x$ then all elements below $y$ are also smaller than $x$, and so $x$ cannot be contained in that column. Therefore the invariant continues to hold with $y$ replaced with its right neighbor (if $y$ has no right neighbors, we can deduce that $x$ is not found in the matrix).

On a $m \times n$ matrix, this process terminates in at most $m+n$ steps, either finding $x$ or determining that it does not appear in the matrix.

\subsolution{Time complexity}

In the worst case, we need to iterate through one row and one column, ending up in the upper right corner element, which means the time complexity is O(m+n).

