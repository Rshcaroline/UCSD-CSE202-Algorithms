\newpage
\problem{5: Frequent elements} % {10+10+10+20=50}

\problemdes

Design an algorithm that, given a list of n elements in an array, finds all the elements that appear more than $n/3$ times in the list. The algorithm should run in linear time. $n$ is a nonnegative integer. 

You are expected to use comparisons and achieve linear time. You may not use hashing. Neither can you use excessive space. Linear space is sufficient. Moreover, you are expected to design a deterministic algorithm. You may also not use the standard linear-time deterministic selection algorithm. Your algorithm has to be more efficient (in terms of constant factors) than the standard linear-time deterministic selection algorithm.

\solution

\subsolution{High-level description}

There can be at most 2 elements that appear more than $n/3$ times. So the size of output will only be $0, 1, 2$. Thus, we maintain two counters to keep record of the most frequent elements. Specifically, we use a dict containing only two elements to realize this goal. When traversing through the array, for element $a_i$, we have the following situations:

1. if $a_i$ is in the dict and $Count_{a_i} > 0$, then $Count_{a_i}+1$; 

2. if $a_i$ is not in the dict:

\indent \indent 2.1 The dict has less than two elements, add $a_i$ into it, and set $Count_{a_i}=1$.

\indent \indent 2.2 The dict has two elements $x, y$, with $Count_{x} > 0$ and $Count_{y} > 0$. Then do $Count_{x} -1$ and $Count_{y} -1$. If after decreasing, we get $Count = 0$. Remove this element from dict.

After traversing through the whole input array, we will get two most frequent elements. Notice! They may not appear more than $n/3$ times. So we need to traverse through the array again and count the appearance of these two elements. Compare the appearance with $n/3$ and then output the result.

\subsolution{Pseudo Code}

In the next page.

\begin{algorithm}[]
  \caption{Frequent elements}
  \KwIn{$A[1 \dots n]$}
  \KwOut{Elements appear more than $n/3$ times in $A$}
  initial dict $D = \{\}$ \;
  \For{$i=1; i \le n; i++$}
  {
    \If{$A[i]$ in dict $D$}
    {
        
        $D[A[i]] = D[A[i]] + 1 $ $//$ Update Count\;
    }
    \Else
    {
        \If{$D$ has less than two elements}
        {
            Add $A[i]$ in $D$ and set $D[A[i]] = 1$\;
        }
        \Else
        {
            \For{element in dict $D$}
            {
                $D[element] = D[element] - 1$\;
                \If{$D[element]$ is $0$}
                {
                    remove $element$ from $D$\;
                }
            }
        }
    }
  }
  let $x, y$ is the two elements of $D$\;
  let $Count_X, Count_Y$ = 0 \;
  \For{$i=1; i \le n; i++$}
  {
    \If{$A[i]$ is $x$}
    {
        $Count_X = Count_X+1$\;
    }
    \If{$A[i]$ is $y$}
    {
        $Count_Y = Count_Y+1$\;
    }
  }
  initial $ret$ as an empty list \;
  \If{$Count_X > n/3$}
  {
    add $x$ to $ret$\;
  }
  \If{$Count_Y > n/3$}
  {
    add $y$ to $ret$\;
  }
  return $ret$\;
\end{algorithm}

\subsolution{Correctness}

There can be at most 2 elements that appear more than $n/3$ times, otherwise will be over $3*n/3=n$ times. So if we want to use voting algorithm, we only need two counters.

We want to prove that the dict stores the most frequent two elements. Say the elements stored in the dict is $a, b$. The first two different elements will be assigned to $a$ and $b$, with $Count_a$ and $Count_b$. We increase the $Count$ when we see $a$ and $b$ is reasonable. The reason why we decrease $Count_a$ and $Count_b$ when the current element is neither $a$ nor $b$ is that if you have some frequent elements, even though you decrease the count when you meet different elements, finally the count will still be larger than 0. You can regard this as a offset of different numbers, only the most frequent ones can survive in the end.

But we can not guarantee its appeared over $n/3$ times. That's why we do another counting for frequent number $a$ and $b$ again.

\subsolution{Time complexity}

The operation in the \textit{For} loop can be completed in $O(1)$ time. Since we traverse through the array, we get a time complexity of $O(n)$.

Also, the space complexity is linear.

\subsolution{Data Structure}

Dict is more efficient for recording the appearance times since you can asses the value of its element by simply using the key in $O(1)$ time. But, considering this specific problem, we only need to record two elements and its appearance times. Four variable is also enough.