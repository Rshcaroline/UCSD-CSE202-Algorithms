\newpage
\problem{2: Remote Sensors} % {10+10+10+20=50}

\problemdes

Devise as efficient as possible algorithm for the following problem. You have n remote sensors $s_i$ and $m < n$ base stations $B_j$. For $1 \le j \le m$, base station $B_j$ is located at $(x_j,y_j)$ in the two-dimensional plane. You are given that no two base-stations are less than 1 km apart (in standard Euclidean distance, $\sqrt{\left(\left(x_{j}-x_{k}\right)^{2}+\left(y_{j}-y_{k}\right)^{2}\right)})$. All base stations have the same integer bandwidth capacity $C$.

For $1 \le i \le n$, sensor $s_i$ is located at $(x_i, y_i)$ in the two-dimensional plane and has an integer bandwidth requirement of $r_i$, which can be met by assigning bandwidth on multiple base stations. Let $b_{i,j}$ be the amount of bandwidth assigned to sensor $s_i$ on base station $B_j$. The assignment must meet the following constraints:

\begin{itemize}
\item No sensor may be assigned any bandwidth on a base station more than 2 km distance from it, i.e., if the distance from $s_i$ to $B_j$ is greater than $2$, $b_{i,j} = 0$.
\item The sum of all the bandwidth assigned to any remote sensor $s_i$ must be at least $r_i$: for each $1 \leq i \leq n$, $\sum_{j} b_{i, j} \geq r_{i}$.
\item The sum of all bandwidth assigned on base station $B_j$ must be at most $C$: for each $1 \leq j \leq m$, $\sum_{i} b_{i, j} \leq C$.
\end{itemize}

Your algorithm should find a solution meeting the above constraints if possible, and otherwise output a message saying “No solution exists”. Prove the correctness of your algorithm and discuss its time complexity.

\solution

\subsolution{High-level description}

We should start by listing key properties of the problem.

\begin{itemize}
\item $n$ remote sensors $s_i$, $\forall i \in \{1, 2, \dots, n\}$, located at $(x_i, y_i)$, with an integer bandwidth requirement of $r_i$ met by assigning bandwidth on multiple base stations
\item $m$ base stations $B_j$, $\forall j \in \{1, 2, \dots, m\}$, located at $(x_j, y_j)$, with the same integer bandwidth capacity $C$
\item No two base stations $B_j$ and $B_k$ are less than 1 km apart
\item No sensors may be assigned any bandwidth on a base station more than 2 km distance from it
\item $b_{i,j}$ be the amount of bandwidth assigned to sensor $s_i$ on base station $B_j$
\item $\forall i \in \{1, 2, \dots, n\}$, $\sum_j b_{i,j} \ge r_i$
\item $\forall j \in \{1, 2, \dots, m\}$, $\sum_i b_{i,j} \le C$
\end{itemize}

We want to find the assignment of bandwidth, which suggests a matching problem, and thus a max-flow problem. We constructed a directed graph $G=(V,E)$.

\begin{itemize}
\item Nodes
	\begin{itemize}
		\item $s$, $t$
		\item $m$ nodes representing the base stations: $B_j$, $\forall j \in \{1, 2, \dots, m\}$
		\item $n$ nodes representing the remote sensors: $s_i$, $\forall i \in \{1, 2, \dots, n\}$
	\end{itemize}
\item Edges
	\begin{itemize}
		\item $s \rightarrow B_{j}$, $\forall j \in \{1, 2, \dots, m\}$, with capacity $C$, which ensures that each base station can have at most $C$ incoming flow
		\item $B_{j} \rightarrow s_i$, $\forall i, j$ where sensors $s_i$ and base stations $B_j$ are less than 2 km apart, with capacity $C$. This ensures that any given base station can only contribute to sensors that it can do.
		\item $s_i \rightarrow t$, $\forall i \in \{1, 2, \dots, n\}$, with capacity $r_{i}$
	\end{itemize}
\end{itemize}

The solution is when the maximum flow in this network $G$ saturates all incoming edges of $t$. Otherwise, the assignment does not exist. As matchings, output the $B_{j} \rightarrow s_i$ flow.

%\subsolution{Pseudo Code}

\subsolution{Correctness}

\begin{itemize}
\item Base stations have incoming flow at most $C$, thus ensuring that the sum of all bandwidth assigned on base station be at most $C$
\item Remote sensors have outgoing flow at most $r_i$, and it has been saturated, so the sum of all the bandwidth assigned to any remote sensor $s_i$ is $r_i$
\item $B_{j} \rightarrow s_i$ edges exist only if the sensors $s_i$ and base stations $B_j$ are less than 2 km apart, so incompatible matches won't arise
\item Each matched base stations contributes a bandwidth of $b_{i,j}$, which ensures the following
	\begin{itemize}
		\item If the $s_i \rightarrow t$ is saturated, then $s_i$ satisfies the bandwidth requirement of $r_i$
		\item If $s_i$ satisfies the bandwidth requirement of $r_i$, then that means the $s_i \rightarrow t$ is saturated
		\item The saturated $s_i \rightarrow t$ edges correctly correspond to the matchings
	\end{itemize}
\end{itemize}

\subsolution{Time complexity}

\begin{itemize}
\item There are $O(n+m)<O(2n)=O(n)$ nodes.
\item There are $O(n)$ edges since no two bases-stations are less than 1 km apart and no sensor may be assigned by any bandwidth on a base station more than 2 km distance from it.
\item With our knowledge of this problem, we see that preflow-push scales better. By using the Preflow-push algorithm to determine the maximum flow, our complexity is $O((n)^2 n)=O(n^3)$.
\end{itemize}








