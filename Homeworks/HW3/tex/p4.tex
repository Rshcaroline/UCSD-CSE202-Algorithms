\newpage
\problem{4: Cellular network} % {10+10+10+20=50}

\problemdes

Consider the problem of selecting nodes for a cellular network. Any number of nodes can be chosen from a finite set of potential locations. We know the cost $c_i \ge 0$ of establishing site $i$. If sites $i$ and $j$ are selected as nodes, then we derive the benefit $b_{ij}$, which is the revenue generated by the traffic between the two nodes. Both the benefits and costs are non-negative integers. Find an efficient algorithm to determine the subset of sites as the nodes for the cellular network such that the sum of the benefits provided by the edges between the selected nodes less the selected node costs is as large as possible.

Design an efficient polynomial-time algorithm.

Provide a high-level description of your algorithm, prove its correctness, and analyze its time complexity.

\solution

\subsolution{High-level description}

Let $n$ be the total number of sites. We constructed a directed graph $G=(V,E)$.

\begin{itemize}
\item Nodes
	\begin{itemize}
		\item $s$, $t$
		\item $n$ nodes representing sites: $S_i$, $\forall i \in \{1, 2, \dots, n\}$
		\item $n^2$ nodes representing benefit: $B_{ij}$, $\forall i \in \{1, 2, \dots, n\}, j \in \{1, 2, \dots, n\}$
	\end{itemize}
\item Edges
	\begin{itemize}
		\item $s \rightarrow B_{ij}$, $\forall i \in \{1, 2, \dots, n\}, j \in \{1, 2, \dots, n\}$, with capacity $b_{ij}$
		\item $B_{ij} \rightarrow S_i$ and $B_{ij} \rightarrow S_j$, $\forall i \in \{1, 2, \dots, n\}, j \in \{1, 2, \dots, n\}$, with capacity $\infty$. We do not want these edges to be cross the cut.
		\item $S_i \rightarrow t$, $\forall i \in \{1, 2, \dots, n\}$, with capacity $c_i$
	\end{itemize}
\end{itemize}

We define an $s$-$t$ cut as a partition $(A, B)$ of $V$ with $s \in A, t \in B$. The solution is: selecting all sites for which $S_i \rightarrow t$ is saturated (i.e., $S_i \in A$).

% \subsolution{Pseudo Code}

\subsolution{Correctness}

\begin{itemize}
\item Consider the types of edges crossing the min-cut. 
	\begin{itemize}
		\item Case 1: $s \rightarrow B_{ij}$: $B_{ij} \in B$. Let $\sum_{ij, \text{where }B_{ij}\notin A} b_{ij}$ be these nodes' contribution to the value of the cut.
		\item Case 2: $S_i \rightarrow t$: $S_i \in A$. Let $\sum_{i, \text{where } S_{i}\in A} c_{i}$ be these nodes' contribution to the value of the cut.
	\end{itemize}
\item Let $C$ be the capacity of the cut. Thus, $C=\sum_{ij, \text{where }B_{ij}\notin A} b_{ij} + \sum_{i, \text{where } S_{i}\in A} c_{i}$. That is, $\sum_{i, \text{where } S_{i}\in A} c_{i} = C - \sum_{ij, \text{where }B_{ij}\notin A} b_{ij} $.
\item $B_{ij}$, for any $i,j$, will only end up in $A$ if both of $S_i,S_j$ are in $A$ (since the intermediate edges have infinite capacity). Thus, our gross benefit is $\sum_{ij, \text{where }B_{ij}\in A}b_{ij}$
\item Our cost is $\sum_{i, \text{where } S_{i}\in A} c_{i}$, as we claimed that these are the sites that we choose.
\item The total profit is 
$$
\begin{aligned}
T&=\sum_{ij, \text{where }B_{ij}\in A}b_{ij}-\sum_{i, \text{where } S_{i}\in A} c_{i}\\
&=\sum_{ij, \text{where }B_{ij}\in A}b_{ij}-(C - \sum_{ij, \text{where }B_{ij}\notin A} b_{ij} )\\
&=\sum_{ij, \text{where }B_{ij}\in A}b_{ij}+ \sum_{ij, \text{where }B_{ij}\notin A} b_{ij} - C\\
&=\sum_{ij, \forall i \in \{1, 2, \dots, n\}, j \in \{1, 2, \dots, n\}}b_{ij} - C\\
\end{aligned}
$$
\item $\sum_{ij, \forall i \in \{1, 2, \dots, n\}, j \in \{1, 2, \dots, n\}}b_{ij}$ is a constant for this network, so we see now that by minimizing this cut, we maximize the profit. Our formulation of the flow network is therefore correct.
\end{itemize}

\subsolution{Time complexity}

\begin{itemize}
\item There are $O(n^2)$ nodes.
\item There are also $O(n^2)$ edges, but we expect the exact number of edges to be greater than the exact number of nodes.
\item $C$ can be arbitrarily large.
\item Because we don’t have a bound on $C$ and the problem specifically asks for a polynomial-time algorithm,
we don’t want something that depends on $C$ . We are thus limited to Edmonds-Karp and preflow-push. With our knowledge of the problem, we see that preflow-push scales better (albeit with the same complexity), so our total complexity is $O((n^2)^2 n^2) = O(n^6)$.  
\end{itemize}




