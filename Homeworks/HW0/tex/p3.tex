\newpage
\problem{3: Maximum difference in a matrix} % {10+10+10+20=50}

\problemdes

Given an $n \times n$ matrix $M[i, j]$ of integers, find the maximum value of $M[c, d]-M[a, b]$ over all choices of indexes such that both $c > a$ and $d > b$.

\solution

\subsolution{High-level description}

Basically Problem 3 shares the same idea with Problem 2: \textit{to record the minimum value $min$ and $maxdifference$ we've ever seen so far}. The differences is that Problem 2 is one dimensional while Problem 3 is two dimensional.

When at a new position $(c, d)$, since the problem description says \textit{the indexes such that both $c>a$ and $d>b$}, so we can only choose the minimum value from $(a,b)$ with the restriction that $1 \le a \le c-1, 1 \le b \le d-1$. Compare $maxdifference$ with $M[c, d]-min$, and update the $maxdifference$ if $M[c, d]-min > maxdifference$.

The trickiest thing is how to find the minimum value from $(a, b)$ efficiently. For $(c,d)$, we have the constriction that $1 \le a \le c-1, 1 \le b \le d-1$. Let's assume the minimum of $M[1 \dots c-1, 1 \dots d-1]$ is $min$. Now let's look at $(c, d+1)$, the constriction is $1 \le a \le c-1, 1 \le b \le d$, which means we should find the minimum of $M[1 \dots c-1, 1 \dots d]$. The interesting part of this problem is: if we already know the minimum of $M[1 \dots c-1, 1 \dots d-1]=min$, to compute $M[1 \dots c-1, 1 \dots d]$, we only need to compare the value of $min$ and the minimum of $M[1 \dots c-1, d]$. The principle of avoid common redundancy will save us lots of time.

We maintain a list $T[1 \dots n]$ to record the minimum of $M[1 \dots c-1, 1 \dots d-1]$. As we iterate by rows, we update it. 

\subsolution{Pseudo Code}

\begin{algorithm}[]
  \caption{Maximum difference in a matrix}
  \KwIn{$M[1 \dots n, 1 \dots n]$}
  \KwOut{$maxdifference$ as maximum difference}
  initial $T[1 \dots n] = \infty$ \;
  let $maxdifference = 0$\;
  \For{$i=1; i \le n-1; i++$}
  {
    $//$ Update $T$ \;
    $T[1] = \min(T[1], M[i, 1])$\;
    \For{$j= 2; j \le n; j++$}
    {
        $T[j] = \min(T[j],  T[j-1], M[i+1, j])$ \;
    }
    $//$ Compare the $min$ with $M[c, d]$ \;
    \For{$j= 2; j \le n; j++$}
    {
        $maxdifference = \max(maxdifference, M[i, j] - T[j-1])$ \;
    }
  }
  return $maxdifference$\;
\end{algorithm}

\subsolution{Correctness}

\item
  Loop Invariant:

  For the \(i\)-\(th\) outer loop, \(T[j]\) is the smallest element in
  \(M[p,q]\) where \(1\le p \le i\) and \(1\le q \le j\)
\item
  Induction:

\begin{enumerate}
  \def\labelenumii{\arabic{enumii}.}
  \item
    Initialization:

    \begin{itemize}
    \item
      When \(n=2\), firstly, we update \(T[1]\) as the smaller one in
      \{\(\infty\) , \(M[1,1]\) \}, so \(T[1]\) satisfies that \(T[j ]\)
      is the smallest element in \(M[p,q]\) where \(1\le p \le 1\) and
      \(1\le q \le 1\) (\(i=j=1 \)).
    \item
      Then when \(j\) increasing from 2, we update
      \(T[j] = \min(T[j], T[j-1], M[i,j])\) , where \(T[j]\) before
      updating is \(\infty\), \(T[j-1]\) is the smallest element in
      \(M[p,q]\) where \(1\le p \le i\) and \(1\le q \le j-1\) . So
      after updating, \(T[j]\) is the smallest element in \(M[p,q]\)
      where \(1\le p \le i\) and \(1\le q \le j\).
    \item
      Therefore, \(T[j]\) is the smallest element in \(M[p,q]\) where
      \(p= i= 1\) and \(1\le q \le j\)
    \end{itemize}
  \item
    Induction:

    \begin{itemize}
    \item
      Assume that For the \(k\)-\(th\) outer loop, \(T[j]\) is the
      smallest element in \(M[p,q]\) where \(1\le p \le k\) and
      \(1\le q \le j\).
    \item
      So for \(k+1\), before updating, \(T[1]\) is the smallest element
      in \(M[p,1]\) where \(1\le p\le k\). When using
      \(T[1]=\min (T[1],M[i,1])\) in the \(k+1\) outer loop, we get
      \(T[1]\) is the smallest in \(M[p,1]\) where \(1\le p \le k+1\).
    \item
      Then, when we update \(T[j] = \min(T[j], T[j-1], M[i,j])\), we
      know \(T[j-1]\) is the smallest in \(M[p,q]\) where
      \(1\le p \le k+1,\ 1\le q \le j-1\). And \(T[j]\) before updating
      is the smallest in \(M[p,q]\) where
      \(1\le p \le k,\ 1\le q \le j\). So after updating with
      \(M[k+1, j]\), it's easy to obtain that \(T[j]\) is the smallest
      element in \(M[p,q]\) where \(1\le p \le k+1\) and
      \(1\le q \le j\).
    \item
      Therefore, \(T[j]\) is the smallest element in \(M[p,q]\) where
      \(1\le p \le i\) and \(1\le q \le j\).
    \end{itemize}
  \end{enumerate}
\item
  Correctness when terminate

  \begin{itemize}
  \item
    Because when \(n\ge i\), this loop will terminate, and \(T[j]\) is
    the smallest element in \(M[p,q]\) where \(1\le p \le n\) and
    \(1\le q \le j \).
  \item
    And we select the maximal difference from the maximal among
    \(M[i,j] - T[j-1]\) in every loop, so we will return maximal
    difference between \(M[i,j]\) and \(M[p,q]\) where \(p<i\) and
    \(q<j\). Therefore, our algorithm is correct.
  \end{itemize}
\end{enumerate}

\subsolution{Time complexity}

$O(n^2)$. Since there are both outer loop of $n$ size and inner loop of $n$ size.

\subsolution{Data Structure}

Two dimensional Array for $M$, and one dimensional array for $T$. We are able to find the element in $O(1)$ time.



